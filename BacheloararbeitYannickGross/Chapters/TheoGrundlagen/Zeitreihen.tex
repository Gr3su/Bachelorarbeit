\section{Zeitreihe}
Eine Zeitreihe ist eine nach den Zeitpunkten aufsteigend sortierte Aneinanderreihung von geordnete Wertepaaren. Dabei besteht ein solches Wertepaar $(t_i, x_i)$ aus einem Zeitpunkt $t_i$ und einem Vektor $x_i$ mit reellen Werten. Die Wertepaare sind Beobachtungen einer Messung, in der ständig in gleichen Abständen ein oder mehrere Werte gleichzeitig erfasst werden. Das heißt, für eine \ac{ZR} gilt 
\begin{equation}
    ZR=[(t_1,x_1),(t_2,x_2),\ldots,(t_n,x_n)], \quad x_i \in \mathbb{R}^m,
\end{equation}
wobei $n$ die Anzahl der Beobachtungen und $m$ die Anzahl der Dimensionen der \acs{ZR} ist. 

Man unterscheidet anhand der Dimension zwei Typen von Zeitreihen. Beträgt die Dimension genau eins, so nennt man sie eine \ac{UZR}, ist es mehr als eine Dimension, so nennt man sie eine \ac{MZR}.

Beispiele für eine \acs{UZR} sind, sind unter anderem der stündliche Verlauf der Temperatur an einer Wetterstation oder der monatliche Umsatz eines Unternehmens, wohingegen die täglichen Eröffnungskurse, Schlusskurse, Tageshochs und Tagestiefs von einer Aktie eine \acs{MZR} mit $m=4$ bilden \cite[Ch. 3.1]{gc2023}.