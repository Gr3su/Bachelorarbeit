\section{Zeitreihe}
Eine Zeitreihe ist eine nach den Zeitpunkten aufsteigend sortierte Aneinanderreihung von geordnete Wertepaaren. Dabei besteht ein solches Wertepaar $(t_i, x_i)$ aus einem Zeitpunkt $t_i$ und einem Vektor $x_i$ mit reellen Werten. Die Wertepaare sind Beobachtungen einer Messung, in der ständig in gleichen Abständen ein oder mehrere Werte gleichzeitig erfasst werden. Das heißt, für eine \ac{ZR} gilt 
\[ZR=[(t_0,x_0),(t_1,x_1),\ldots,(t_{n-1},x_{n-1})], \quad x_i \in \mathbb{R}^d,\]
wobei $n$ die Anzahl der Beobachtungen und $d$ die Anzahl der Dimensionen der \acs{ZR} ist.\footnote{In der gesamten Arbeit wird durchgehend die Indizierung von 0 bis $n-1$ verwendet, da dies der Implementierung in der benutzten Programmiersprache Python entspricht.}\label{foot:indexe} Eine Dimension wird in diesem Kontext auch Feature genannt, da in einer Dimension nur Werte eines bestimmten Typs (Temperatur, Feuchtigkeit, Windstärke, \dots) gespeichert werden.

Man unterscheidet anhand der Dimensionalität zwei Typen von Zeitreihen. Beträgt die Dimensionalität genau eins, so nennt man sie eine \ac{UZR}, ist es mehr als eine Dimension, so nennt man sie eine \ac{MZR}.

Beispiele für eine \acs{UZR} sind unter anderem der stündliche Verlauf der Temperatur an einer Wetterstation oder der monatliche Umsatz eines Unternehmens, wohingegen die täglichen Eröffnungskurse, Schlusskurse, Tageshochs und Tagestiefs von einer Aktie eine \acs{MZR} mit $d=4$ bilden \cite[Ch. 3.1]{compressionSurvey}.

In unserer Anwendung werden wir mehrere \acs{UZR} zu einer gemeinsamen \acs{MZR} zusammenführen. Für die im Experiment verwendeten Bibliotheken muss die Struktur der übergebenen Daten allerdings anders aussehen. Dafür befindet sich in jeder Spalte eine \acs{UZR} und die Zeilen stellen die Zeitpunkte dar.