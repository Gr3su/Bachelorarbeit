\chapter{Implementierungen}
\begin{wrapfigure}{r}{3cm}
    \centering
    \qrcode[height=2.5cm]{https://github.com/Gr3su/Bachelorarbeit}
\end{wrapfigure}

In diesem Kapitel wird die Implementierung verschiedener zentraler Funktionen vorgestellt, mit denen das Experiment in \autoref{ch:experimentundergebnis} durchgeführt wurde, darunter solche zur Kompression, zur Anomalieerkennung sowie zur Ähnlichkeitssuche. Das dazugehörige GitHub Repository, in dem der vollständige Code zu finden ist, ist entweder über die Quelle \cite{g2025} oder über den QR-Code am rechten Rand zu finden. Außerdem ist ein \textit{Dateiname} in diesem Kapitel kursiv geschrieben, wenn sich auf eine Datei aus dieser Quelle bezogen wird. 

\input{Chapters/Implementierungen/NützlicheFunktionen.tex}
\section{Kompressions Algorithmen}\label{sec:kompressionsAlgorithmen}
Alle Funktionen in diesem Kapitel haben an ihrem Anfang den Aufruf der Funktion \lstinline|checkParameters|, um die Datentypen der übergebenen Argumente zu validieren. In diesem Teil des Codes kommen zwei Open Source Bibliotheken zum Einsatz, und zwar \lstinline|Numpy| \cite{cksr2020} und \lstinline|PyWavelets| \cite{grfk2019}.

Die Funktion \lstinline|polynomialApproximation| in \textit{CompressionAlgorithms.py}, auch zu sehen in \autoref{lst:stückweisepolynomielleapproximation}, ist dazu da, jede Variable einer \ac{MZR} stückweise polynomielle zu approximieren und dadurch zu komprimieren. Die Funktion \lstinline|linearApproximation| ist identisch zu dieser, außer dass der Parameter \lstinline|deg| in Zeile 9 auf 1 statt 3 gesetzt wird. Daher wird diese Funktion in \autoref{sec:kompressionsAlgorithmen} nicht weiter beschrieben.

In Zeile 4 wird die \lstinline|list| Variable \lstinline|compressedMultivariateTimeSeries| initialisiert, in der die neuen \lstinline|list| Objekte gespeichert werden, die die komprimierte Werte enthalten. Die erste \lstinline|for|"=Schleife iteriert über die Variablen bzw. die einzelnen Zeitreihen der \ac{MZR}, dabei werden in \lstinline|coefficients| die Koeffizienten der aktuellen Zeitreihe gespeichert. In der zweiten \lstinline|for|"=Schleife werden die einzelnen Segmente approximiert. Dafür iteriert die Schleife über die Indexe der Zeitreihe, die die linken Ränder der Segmente sind und berechnet in Zeile 8 dann den rechten Rand, indem auf den Wert des linken Randes die Segmentlänge \lstinline|segmentLength| addiert wird. Liegt der berechnete rechte Rand oberhalb der Länge der Zeitreihe, so wird er auf den letzten möglichen Wert \lstinline|len(i)| gesetzt. Um der Konvention in Python gerecht zu werden, ist der linke Rand des Segments inklusiv und der rechte Rand exklusiv. Somit kann in Zeile 9 mit \lstinline|i[j:rightBoundary]| direkt das Segment aus der ursprünglichen Zeitreihe herausgeschnitten werden. Die Funktion \lstinline|fit| aus \lstinline|numpy.polynomial.polynomial.Polynomial| findet mit dem Übergabeparameter \lstinline|deg=3| das Polynom 3. Grades, das den least squares fit zu dem Segment \lstinline|i[j:rightBoundary]| hat. Um interne umskalierungen von \lstinline|numpy| rückgängig zu machen und um dann die Koeffizienten des Polynoms zu erhalten, wird in Zeile 10 \lstinline|p.convert().coef| aufgerufen. Diese Koeffizienten werden dann der Liste mit den anderen Koeffizienten angehangen, um nach Beendigung der zweiten Schleife als Liste zu \lstinline|compressedMultivariateTimeSeries| hinzugefügt zu werden.
\begin{lstlisting}[caption=Stückweise polynomielle Approximation, label=lst:stückweisepolynomielleapproximation, style=Python, language=Python]
def polynomialApproximation(multivariateTimeSeries : mts, segmentLength : int):
    checkParameters(multivariateTimeSeries, segmentLength)

    compressedMultivariateTimeSeries = []
    for i in multivariateTimeSeries.multivariateTimeSeries:
        coefficients = np.array([])
        for j in range(0, multivariateTimeSeries.timeSeriesLength, segmentLength):
            rightBoundary = j + segmentLength if j + segmentLength <= len(i) else len(i)
            p = Pol.fit(range(0, segmentLength), i[j:rightBoundary], deg=3)
            coefficients = np.append(coefficients, p.convert().coef)
        
        compressedMultivariateTimeSeries.append(coefficients.astype(float).tolist())
    return compressedMultivariateTimeSeries
\end{lstlisting}

Die Funktion \lstinline|dwtApproximation| in \textit{CompressionAlgorithms.py}, auch zu sehen in \autoref{lst:diskretewavelettransformation}, ist dazu da, jede Variable einer \ac{MZR} mit der diskreten Wavelet"=Transformation zu approximieren und dadurch zu komprimieren.

In Zeile 4 wird die \lstinline|list| Variable \lstinline|compressedMultivariateTimeSeries| initialisiert, in der die neuen \lstinline|list| Objekte gespeichert werden, die die komprimierte Werte enthalten. Die erste \lstinline|for|"=Schleife iteriert über die Variablen bzw. die einzelnen Zeitreihen der \ac{MZR}, dabei wird in \lstinline|data| der aktuelle Stand der Kompression gespeichert. In Zeile 8 werden dann mittels der Funktion \lstinline|pywt.dwt| und der Wavelet \lstinline|db1| die Daten approximiert. Die Approximations"= und Differenz"=Koeffizienten sind die Rückgabewerte der Funktion, wobei \lstinline|data| nur mit den Approximations"=Koeffizienten überschrieben wird. 
\begin{lstlisting}[caption=Stückweise polynomielle Approximation, label=lst:diskretewavelettransformation, style=Python, language=Python]
def dwtApproximation(multivariateTimeSeries : mts, iterations : int):
    checkParameters(multivariateTimeSeries, iterations)

    compressedMultivariateTimeSeries = []
    for i in multivariateTimeSeries.multivariateTimeSeries:
        data = i
        for j in range(0, iterations):
            coefApprox, coefDiff = pywt.dwt(data, 'db1')
            data = coefApprox
        compressedMultivariateTimeSeries.append(data.astype(float).tolist())
    return compressedMultivariateTimeSeries
\end{lstlisting}
\section{Algorithmen zur Ausreißererkennung}
Die Funktionen \lstinline|knnDetection|, \lstinline|isolationForestDetection| und \texttt{randomProjectionsDetec\allowbreak tion} in \textit{AnomalyAlgorithms.py}, auch zu sehen in \autoref{lst:pyodAlgorithms} beziehungsweise \autoref{lst:randomProjection}, ist dazu da, unter den Zeitreihen einer \ac{MZR} Ausreißer zu finden. Dafür wird die Open Source Bibliothek \lstinline|PyOD| (Python Outlier Detection) V2 \cite{pyod} genutzt. \lstinline|PyOD| ermöglicht es, durch ein sehr simples, objektorientiertes Interface, die implementierten Ausreißererkennungs"=Verfahren zu nutzen. Mit \lstinline|KNN()| und \lstinline|IForest()| in \autoref{lst:pyodAlgorithms} bzw. mit \lstinline|MAD()| in \autoref{lst:randomProjection}, aus den jeweiligen Klassen in \lstinline|pyod.models|, wird der Standardkonstruktor der Detektoren aufgerufen, dieser setzt gewisse Parameter wie die \lstinline|contamination| oder den \lstinline|threshold| auf ihre Standardwerte. Mit dem Aufruf der Methode \lstinline|fit| wird der Detektor mit den Daten der \ac{MZR} geladen und der jeweilige Algorithmus verarbeitet dann diese Daten, um die Zeitreihen nach ihren Auffälligkeiten zu bewerten. Umso höher diese Wertung ist, umso eher ist eine Zeitreihe als Ausreißer zu werten. Die Verfahren \lstinline|knn| und \lstinline|iForest| bekommen im Standardkonstruktor den Wert \lstinline|contamination=0.1| gesetzt, das bedeutet, dass 10~\% der Zeitreihen als Ausreißer markiert werden, und zwar die, mit den höchsten Wertungen. Im Attribut \lstinline|labels_| findet sich nach dem Aufruf von \lstinline|fit| ein Array der Länge entsprechend der Anzahl an Zeitreihen in der übergebenen \ac{MZR}. Dieses Array enthält für jede Zeitreihe das Ergebnis der Ausreißererkennung: 0 für Normal, 1 für Ausreißer.
\begin{lstlisting}[caption=Stückweise polynomielle Approximation, label=lst:pyodAlgorithms, style=Python, language=Python]
def knnDetection(multivariateTimeSeries : mts):
    detector = KNN()
    detector.fit(multivariateTimeSeries.multivariateTimeSeries)
    return detector.labels_

def isolationForestDetection(multivariateTimeSeries : mts):
    detector = IForest()
    detector.fit(multivariateTimeSeries.multivariateTimeSeries)
    return detector.labels_
\end{lstlisting}

In \autoref{lst:randomProjection} ist es etwas mehr Code, da \lstinline|PyOD| dieses Verfahren nicht direkt implementiert. Die \lstinline|scores| Variable speichert für jede Zeitreihe die addierten Wertungen der Wahrscheinlichkeit eines Ausreißers. Die erste \lstinline|for|"=Schleife erstellt einen Vektor mit zufälligen positiven Werten, der dann in der zweiten \lstinline|for|"=Schleife folgendermaßen mit jeder Zeitreihe verknüpft wird: \\ \lstinline[breaklines=true]|randomV[0] * series[0] + randomV[1] * series[1] +| \dots{}\lstinline|+ randomV[-1] * series[-1]|. \\ Daraufhin werden die erhaltenen Werte Mittels des Detektors \lstinline|MAD| auf Ausreißer untersucht und die Wertungen in der \lstinline|scores| Variable gespeichert. Nach Beendigung der ersten Schleife wird unter den Scores, wieder mit \lstinline|MAD|, nach Ausreißern gesucht.
\begin{lstlisting}[caption=Stückweise polynomielle Approximation, label=lst:randomProjection, style=Python, language=Python]
def randomProjectionsDetection(multivariateTimeSeries : mts):
    scores = np.zeros(len(multivariateTimeSeries.multivariateTimeSeries))

    for i in range(100):
        randomVector = np.abs(np.random.randn(multivariateTimeSeries.timeSeriesLength))
        values = []
        for series in multivariateTimeSeries.multivariateTimeSeries:
            values.append([np.dot(series, randomVector)])
        detector = MAD()
        detector.fit(values)
        scores += detector.decision_scores_
    detector = MAD()
    detector.fit([[x] for x in scores])
    return detector.labels_
\end{lstlisting}