\chapter{Implementierungen}\label{ch:implementierung}
\begin{wrapfigure}{r}{3cm}
    \centering
    \qrcode[height=2.5cm]{https://github.com/Gr3su/Bachelorarbeit}
\end{wrapfigure}

In diesem Kapitel wird die Implementierung verschiedener zentraler Funktionen vorgestellt, mit denen das Experiment in \autoref{ch:experimentundergebnis} durchgeführt wurde, darunter solche zur Kompression, zur Anomalieerkennung sowie zur Ähnlichkeitssuche. Das dazugehörige GitHub"=Repository, in dem der vollständige Code zu finden ist, ist entweder über die Quelle \cite{meinGithubCode} oder über den QR-Code am rechten Rand zu finden. Außerdem sind \texttt{Dateiname} und \texttt{Code} in diesem Kapitel in Schreibmaschinenschrift geschrieben, um diese klar zu kennzeichnen. 

\section{Nützliche Funktionen}
Die Klasse MultivariateTimeSeries in \textit{Utilities.py}, auch zu sehen in \autoref{lst:multivariateTimeSeries}, ist dazu da eine \ac{MZR} abzuspeichern und zu validieren.

Die Methode \lstinline{__init__} ist der Konstruktor, der eine \ac{MZR} entgegennimmt und diese in den Zeilen 12 bis 15 auf die richtigen Datentypen validiert. In Zeile 17 wird das erste Element aus \lstinline|timeSeries| zum Attribut \lstinline|multivariateTimeSeries| hinzugefügt und in Zeile 18 wird die Anzahl der Elemente 

\begin{lstlisting}[caption=Klasse für \ac{MZR}, language=Python, label=lst:multivariateTimeSeries, style=Python]
class MultivariateTimeSeries:
    def addTimeSeries(self, timeSeries):
        if not isinstance(timeSeries, list) and not isinstance(timeSeries, np.ndarray):
            raise TypeError(f"Erwarte eine List, aber ein {type(timeSeries)} erhalten.")
        
        if len(timeSeries) != self.timeSeriesLength:
            raise ValueError(f"Das Segment hat {len(timeSeries)} Element(e), muss aber {self.timeSeriesLength} haben.")
        
        self.multivariateTimeSeries.append(timeSeries)
    
    def __init__(self, timeSeries):
        if not isinstance(timeSeries, list):
            raise TypeError(f"Erwarte eine List, aber ein {type(timeSeries)} erhalten.")
        if not isinstance(timeSeries[0], list) and not isinstance(timeSeries[0], np.ndarray):
            raise TypeError(f"Erwarte eine List, aber ein {type(timeSeries[0])} erhalten.")
        
        self.multivariateTimeSeries = [timeSeries[0]]
        self.timeSeriesLength = len(timeSeries[0])
        for i in range(1,len(timeSeries)):
            self.addTimeSeries(timeSeries[i])
\end{lstlisting}
\section{Kompressions Algorithmen}\label{sec:kompressionsAlgorithmen}
Alle Funktionen in diesem Kapitel haben an ihrem Anfang den Aufruf der Funktion \lstinline|checkParameters|, um die Datentypen der übergebenen Argumente zu validieren. In diesem Teil des Codes kommen zwei Open Source Bibliotheken zum Einsatz, und zwar \lstinline|Numpy| \cite{cksr2020} und \lstinline|PyWavelets| \cite{grfk2019}.

Die Funktion \lstinline|polynomialApproximation| in \textit{CompressionAlgorithms.py}, auch zu sehen in \autoref{lst:stückweisepolynomielleapproximation}, ist dazu da, jede Variable einer \ac{MZR} stückweise polynomielle zu approximieren und dadurch zu komprimieren. Die Funktion \lstinline|linearApproximation| ist identisch zu dieser, außer dass der Parameter \lstinline|deg| in Zeile 9 auf 1 statt 3 gesetzt wird. Daher wird diese Funktion in \autoref{sec:kompressionsAlgorithmen} nicht weiter beschrieben.

In Zeile 4 wird die \lstinline|list| Variable \lstinline|compressedMultivariateTimeSeries| initialisiert, in der die neuen \lstinline|list| Objekte gespeichert werden, die die komprimierte Werte enthalten. Die erste \lstinline|for|"=Schleife iteriert über die Variablen bzw. die einzelnen Zeitreihen der \ac{MZR}, dabei werden in \lstinline|coefficients| die Koeffizienten der aktuellen Zeitreihe gespeichert. In der zweiten \lstinline|for|"=Schleife werden die einzelnen Segmente approximiert. Dafür iteriert die Schleife über die Indexe der Zeitreihe, die die linken Ränder der Segmente sind und berechnet in Zeile 8 dann den rechten Rand, indem auf den Wert des linken Randes die Segmentlänge \lstinline|segmentLength| addiert wird. Liegt der berechnete rechte Rand oberhalb der Länge der Zeitreihe, so wird er auf den letzten möglichen Wert \lstinline|len(i)| gesetzt. Um der Konvention in Python gerecht zu werden, ist der linke Rand des Segments inklusiv und der rechte Rand exklusiv. Somit kann in Zeile 9 mit \lstinline|i[j:rightBoundary]| direkt das Segment aus der ursprünglichen Zeitreihe herausgeschnitten werden. Die Funktion \lstinline|fit| aus \lstinline|numpy.polynomial.polynomial.Polynomial| findet mit dem Übergabeparameter \lstinline|deg=3| das Polynom 3. Grades, das den least squares fit zu dem Segment \lstinline|i[j:rightBoundary]| hat. Um interne umskalierungen von \lstinline|numpy| rückgängig zu machen und um dann die Koeffizienten des Polynoms zu erhalten, wird in Zeile 10 \lstinline|p.convert().coef| aufgerufen. Diese Koeffizienten werden dann der Liste mit den anderen Koeffizienten angehangen, um nach Beendigung der zweiten Schleife als Liste zu \lstinline|compressedMultivariateTimeSeries| hinzugefügt zu werden.
\begin{lstlisting}[caption=Stückweise polynomielle Approximation, label=lst:stückweisepolynomielleapproximation, style=Python, language=Python]
def polynomialApproximation(multivariateTimeSeries : mts, segmentLength : int):
    checkParameters(multivariateTimeSeries, segmentLength)

    compressedMultivariateTimeSeries = []
    for i in multivariateTimeSeries.multivariateTimeSeries:
        coefficients = np.array([])
        for j in range(0, multivariateTimeSeries.timeSeriesLength, segmentLength):
            rightBoundary = j + segmentLength if j + segmentLength <= len(i) else len(i)
            p = Pol.fit(range(0, segmentLength), i[j:rightBoundary], deg=3)
            coefficients = np.append(coefficients, p.convert().coef)
        
        compressedMultivariateTimeSeries.append(coefficients.astype(float).tolist())
    return compressedMultivariateTimeSeries
\end{lstlisting}

Die Funktion \lstinline|dwtApproximation| in \textit{CompressionAlgorithms.py}, auch zu sehen in \autoref{lst:diskretewavelettransformation}, ist dazu da, jede Variable einer \ac{MZR} mit der diskreten Wavelet"=Transformation zu approximieren und dadurch zu komprimieren.

In Zeile 4 wird die \lstinline|list| Variable \lstinline|compressedMultivariateTimeSeries| initialisiert, in der die neuen \lstinline|list| Objekte gespeichert werden, die die komprimierte Werte enthalten. Die erste \lstinline|for|"=Schleife iteriert über die Variablen bzw. die einzelnen Zeitreihen der \ac{MZR}, dabei wird in \lstinline|data| der aktuelle Stand der Kompression gespeichert. In Zeile 8 werden dann mittels der Funktion \lstinline|pywt.dwt| und der Wavelet \lstinline|db1| die Daten approximiert. Die Approximations"= und Differenz"=Koeffizienten sind die Rückgabewerte der Funktion, wobei \lstinline|data| nur mit den Approximations"=Koeffizienten überschrieben wird. 
\begin{lstlisting}[caption=Stückweise polynomielle Approximation, label=lst:diskretewavelettransformation, style=Python, language=Python]
def dwtApproximation(multivariateTimeSeries : mts, iterations : int):
    checkParameters(multivariateTimeSeries, iterations)

    compressedMultivariateTimeSeries = []
    for i in multivariateTimeSeries.multivariateTimeSeries:
        data = i
        for j in range(0, iterations):
            coefApprox, coefDiff = pywt.dwt(data, 'db1')
            data = coefApprox
        compressedMultivariateTimeSeries.append(data.astype(float).tolist())
    return compressedMultivariateTimeSeries
\end{lstlisting}
\section{Algorithmen zur Ausreißererkennung}
Die Funktionen \lstinline|knnDetection| und \lstinline|isolationForestDetection| in \textit{AnomalyAlgorithms.py}, auch zu sehen in \autoref{lst:pyodAlgorithms}, ist dazu da, unter den Zeitreihen einer \ac{MZR} Ausreißer zu finden. Dafür wird die Open Source Bibliothek \lstinline|PyOD| (Python Outlier Detection) V2 \cite{zhao2024} genutzt. \lstinline|PyOD| ermöglicht es, durch ein sehr simples, objektorientiertes Interface, die implementierten Ausreißererkennungs"=Verfahren zu nutzen. 
\begin{lstlisting}[caption=Stückweise polynomielle Approximation, label=lst:pyodAlgorithms, style=Python, language=Python]
def knnDetection(multivariateTimeSeries : mts):
    clf = KNN()
    clf.fit(multivariateTimeSeries.multivariateTimeSeries)
    return clf.labels_

def isolationForestDetection(multivariateTimeSeries : mts):
    clf = IForest()
    clf.fit(multivariateTimeSeries.multivariateTimeSeries)
    return clf.labels_
\end{lstlisting}

\begin{lstlisting}[caption=Stückweise polynomielle Approximation, label=lst:randomProjection, style=Python, language=Python]
def randomProjectionsDetection(multivariateTimeSeries : mts):
    scores = np.zeros(len(multivariateTimeSeries.multivariateTimeSeries))

    for i in range(100):
        randomVector = np.random.randn(multivariateTimeSeries.timeSeriesLength)
        coeffs = []
        for series in multivariateTimeSeries.multivariateTimeSeries:
            coeffs.append([np.dot(series, randomVector)])
        clf = MAD()
        clf.fit(coeffs)
        scores += clf.decision_scores_
    clf = MAD()
    clf.fit([[x] for x in scores])
    return clf.labels_
\end{lstlisting}