\section{Aufbau der Arbeit}
Der weitere Verlauf dieser Arbeit ist wie folgt aufgebaut: in \autoref{ch:theoretischegrundlagen} werden die verschiedenen Begriffe, wie Zeitreihe, Ausreißer, Ähnlichkeit und Kompression, erläutert, sowie die verschiedenen Verfahren zur Kompression und Analyse der Zeitreihen. \autoref{ch:methodik} stellt die benutzten Datensätze vor und erklärt wie diese für das Experiment vorbereitet wurden. Außerdem wird darauf eingegangen welche Parameter zur Kompression und Analyse benutzt wurden. In \autoref{ch:implementierung} wird auf wichtige Teile des Codes eingegangen und dessen Funktionsweise erklärt. \autoref{ch:experimentundergebnis} präsentiert letztendlich die Ergebnisse des Experiments und \autoref{ch:fazit} schließt die Arbeit mit einem Fazit ab.