\section{Motivation}

Zeitreihendaten sind die Grundlage für zahlreiche Anwendungen in Wissenschaft, Technik und Wirtschaft. Sie entstehen etwa bei der Überwachung von Maschinenzuständen, der Aufzeichnung medizinischer Messungen oder der Analyse von Finanzmärkten. Häufig werden diese Daten dann gespeichert, verarbeitet oder übertragen, was bei großen Mengen oder begrenzten Ressourcen unterschiedliche Herausforderungen mit sich bringt: so zum Beispiel Speicherplatzverschwendung, die Dauer von Übertragungen oder der Bearbeitungszeit.

Eine Möglichkeit, diesen Herausforderungen zu entgegnen, ist die Kompression der Zeitreihen. Dabei stellt sich aber die Frage, wie sich die Kompressionsverfahren auf die Qualität späterer Analysen auswirken. Insbesondere die Auswirkung auf die Ähnlichkeitssuche und der Ausreißererkennung ist interessant, da diese in vielen Anwendungen die wichtigste Operation sind. Werden bei der Kompression relevante Merkmale der Zeitreihe verfälscht oder gehen sogar verloren, können Ähnlichkeitsmaße verzerrt oder Anomalien übersehen werden.

Die Motivation dieser Arbeit liegt darin, zu untersuchen, ob sich verschiedene Analyseverfahren auf den komprimierten Daten anwenden lassen, und wenn ja, welche Auswirkungen die Kompressionsverfahren auf die Analyse haben. Ziel soll es also sein, einen Einblick darin zu bekommen, inwieweit sich die Komprimierung von Zeitreihendaten mit der zuverlässigen Durchführung von Ähnlichkeitssuchen und der Detektion von Ausreißern vereinbaren lässt. Diese Fragestellung hat in der Praxis Relevanz, da in nahezu allen Anwendungsfällen effizientes Speichern mit einer möglichst präzisen Analyse vereinbart werden muss -- beispielsweise bei der Überwachung von Industrieanlagen, Rechenzentren oder bei der Verarbeitung von Sensordaten im Internet"=of"=Things"=Bereich.