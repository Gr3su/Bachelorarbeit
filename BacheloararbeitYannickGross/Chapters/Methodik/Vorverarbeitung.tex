\section{Vorverarbeitung}
Die Datensätze müssen für die Weiterverarbeitung nach dem Herunterladen vorbereitet werden, denn sie enthalten nicht nur die Werte, die wir brauchen, sondern auch Zeitstempel, Beschreibungen der Daten und weitere Zusätze. Um die Nachprüfbarkeit des Experiments zu erleichtern und um selbst schneller die Daten nutzen zu können, übernimmt das Skript, das zum Teil in \autoref{lst:dataPreparation} erwähnt wird, diese Aufgabe.

Aufgabe dieses Skripts ist allgemein nur die Werte herauszufiltern, die für das Experiment von Relevanz sind. Das wäre beim Aktienkurs die Kursdifferenz und bei den Wetterdaten die Durchschnittstemperatur. Der EKG"=Datensatz beinhaltet bereits lediglich die Messwerte. 

Des Weiteren nimmt das Skript noch weitere, spezifischere Schritte vor. Die Zeitreihe des Aktienkurses wird in weitere Zeitreihen mit nur 50 Werten aufgeteilt, so dass später der Aktienkurs in Abschnitten von 50 Tagen analysiert werden kann. Diese neuen Zeitreihen werden dann in Dateien mit den Namen \textit{50.csv, 100.csv, 150.csv, \dots} geschrieben. Die Wetterdaten enthalten für jeden Tag einen \textit{Quality Code}, der entweder 0 (valide), 1 (auffällig) oder 9 (fehlt) ist. Wetterstationen, die an einem oder mehreren Tagen einen \textit{Quality Code} ungleich 0 haben, werden komplett ignoriert und nicht weiter verwendet. Unter den übrig gebliebenen werden die aussortiert, die weniger als 1.000 Messwerte haben, da die letzten 1.000 Tage jeder Wetterstation zur späteren Analyse verwendet werden. Die neuen Zeitreihen haben dabei denselben Dateinamen wie die vorherigen, lediglich die Endung ändert sich zu \text{.csv}. Bei den EKG"=Daten befinden sich alle Herzschläge in einer Zeitreihe, für unser Vorhaben müssen diese allerdings, wie der Aktienkurs auch, in mehrere aufgeteilt werden. So wird jeder Herzschlag, also 140 Werte, in eine neue Zeitreihe gepackt. Diese neuen Zeitreihen werden dann in Dateien mit den Namen \textit{140.csv, 280.csv, 420.csv, \dots} geschrieben.