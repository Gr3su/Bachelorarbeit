\section{Vorverarbeitung}
Die Datensätze müssen für die Weiterverarbeitung nach dem Herunterladen vorbereitet werden, denn sie enthalten nicht nur die Werte, die wir brauchen, sondern auch Zeitstempel, Beschreibungen der Daten und weitere Zusätze. Um die Nachprüfbarkeit des Experiments zu erleichtern und um selber schneller die Daten nutzen zu können, übernimmt das Skript, das zum Teil in \autoref{lst:dataPreparation} erwähnt wird, diese Aufgabe.

Aufgabe dieses Skript ist allgemein nur die Werte herauszufiltern, die für das Experiment von Relevanz sind. Das wäre beim Aktienkurs die Kursdifferenz und bei den Wetterdaten die Durchschnittstemperatur. Der EKG"=Datensatz beinhaltet bereits lediglich die Messwerte. Des Weiteren nimmt das Skript noch weitere, spezifischere Schritte vor. So wird die Zeitreihe des Aktienkurses in weitere Zeitreihen mit nur 50 Werten aufgeteilt, sodass später der Aktienkurs in Abschnitten von 50 Tagen analysiert werden kann. Die Wetterdaten enthalten für jeden Tag einen \textit{Quality Code}, der entweder 0 (valide), 1 (auffällig) oder 9 (fehlt) ist.