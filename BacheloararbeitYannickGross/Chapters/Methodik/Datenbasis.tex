\section{Datenbasis}
Für das Experiment wurden drei verschiedene Datensätze benutzt, die verschiedene Fälle abdecken: 
\begin{itemize}
    \item der Aktienkurs von NBIDIA \cite{nvidiaStock}, wobei hier die Differenz zwischen Schlusskurs und Eröffnungskurs genutzt wird,
    \item die tägliche Durchschnittstemperatur an Wetterstationen in Europa \cite{ecadWetterdaten},
    \item ein EKG \cite{ecg500}, wobei hier der Datensatz \textit{ECG5000\_TRAIN.txt} genutzt wird, der 500 Herzschläge beinhaltet.
\end{itemize}

Aktienkurse sind interessant, da man in ihnen Trends erkennen kann oder sogar Saisonalität, also z.~B. dass der Kurs zum Jahresende immer steigt, weil der Umsatz wegen Weihnachten steigt. Ansonsten ist die Struktur der Daten unorganisiert, heißt die Daten rauschen viel und es kann extreme Sprünge geben.

Die Wetterdaten stammen von verschiedenen Wetterstationen aus ganz Europa und sollen dadurch verschiedene Klimazonen abdecken. Wegen der Jahreszeiten lassen sich somit die Daten gut bezüglich ihrer Saisonalität vergleichen, besser als bei Aktienkursen. Allerdings sind weniger Trends zu erkennen, da diese über längere Perioden geschehen, als wir sie betrachten. Die Struktur von Wetterdaten ist sehr gleichförmig, bis auf die Schwankungen zwischen den Jahreszeiten sind die Verläufe sehr stetig.

Die EKG"=Daten wiederum sind interessant, weil sie weder Saisonalität noch Trends bieten, dafür eine sehr strenge Struktur haben. Sie wiederholen sich in einer sehr kleinen Periode und haben eine feste Wellenform, von der nur bei bestimmten Ereignissen abgewichen wird.